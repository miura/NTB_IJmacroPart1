\textbf{sourcecode} : \href{http://www.example.com/contents}{code/code12.ijm}

\begin{itemize}
\item Line 3 The macro asks user to input a number and the number is substituted to the variable input\_num.
\item Line 4 Content of input\_num is evaluated. If input\_num is equal to 5, line 5 is executed and prints out the message in the Log window. Otherwise macro interpreter jumps to line 7, and ends the operation.  By adding "else" which will be executed if input\_num is not 5, the macro prints out message in all cases (see code 12.5 for this if - else case). 
\item Line 4 We used double equal signs for evaluating the value in the right side and the left side (e.g. \ilcom{if (a==5)}). 
Note that the role of the sign \ilcom{=} is different from assignments, or substitution (e.g. \ilcom{a = b + c}).

\end{itemize}