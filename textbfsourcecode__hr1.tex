
\begin{indentexercise}
{2}
\item Try modifying the third line in code 1.51
and check that the modified text will be printed in the "Log" window. \\

\end{indentexercise}

\begin{indentexercise}
{3}
\item Multiple macros can exist in a single file. We call this \textbf{"macro sets"} . To distinguish each macro, each they each should have a specific name. For this, each macro should start with a special word ``macro'' followed by the name of the macro, and then a pair of curly braces to encapsulate its macro functions. See the code below (code01\_8.ijm).

Modify the code you already wrote in the script editor to wrap it inside a pair of macro bounds, the curly braces (\ilcom{\{\}}).  Then copy and paste the same under the first macro. 
The second macro should be modified to have a different name. In the example shown in fig.
\ref{fig_MacroSetInMenu}, the second macro is named "print\_out2".

When macro is properly declared in this way, you could install the macro to have it as a menu item. To do so, in the editor menu select:

\begin{indentFiji}
[Run -> Install Macro]).

\end{indentFiji}
In the main menu you should no be able to see the macro names under \ijmenu{[Plugins > Macros > ]}. See fig. \ref{fig_MacroInMenu}}

\end{indentexercise}