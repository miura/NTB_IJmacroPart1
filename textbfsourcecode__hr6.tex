\textbf{sourcecode} : \href{http://www.example.com/contents}{code/code07.ijm}

The third line in the above macro has a function \ilcom{newImage()}. This
function creates a new image. It has five arguments (in coding jargon, we say
there are "five arguments"). To know what these arguments are, 
the quickest way is to read the Build-In Macro Function page in ImageJ web site\footnote{\url{http://rsbweb.nih.gov/ij/developer/macro/functions.html}}.  
The description of the function \ilcom{newImage} looks like this.

\begin{indentCom}

\fbox{
\parbox[b][16em][c]{0.80\textwidth}{
\textbf{newImage} (title, type, width, height, depth)\\
Opens a new image or stack using the name title. 
The string type should contain "8-bit", "16-bit", "32-bit" or "RGB". 
In addition, it can contain "white", "black" or "ramp" (the default is "white"). 
As an example, use "16-bit ramp" to create a 16-bit image containing a grayscale ramp.  Width and height specify the width and height of the image in pixels.  Depth specifies the number of stack slices.
}}

\end{indentCom}
Using this information, you can modify the macro to change the size of the image.

\begin{indentexercise}
{1}
Modify the code 7 and try changing the size of window to be created.

\item \textbf{Answer} : Change the 3rd line as shown below. It will create a image with 500 pixels width and 200 pixels height image.