\section{Basics}
\label{sec:ImageJMacroBasics}

\subsection{``Hello World!''}
We first try writing a simple macro that prints ''Hello World!'' in the log window of ImageJ. For this, we use a text editor that comes with Fiji, called ``script editor''. 
It has some convenient features such as automatic coloring of macro functions {In programming world, we call this feature ``syntax highlighter''}.

If you are not using Fiji and using native ImageJ, it's no problem as there is a simpler but perfectly working text editor. Macros we write in this textbook works exactly same in both editors. If you want to use the simple text editor for the following tutorial, its usage is explained after the explanation about Fiji edior (page \pageref{part:nativeeditor}).

Let's open the ``script editor'' 
by \ijmenu{[File -> New -> Script]}. It should look like figure \ref{fig_ScriptEditor}.
There should be a blank text field where you write your macro. Since the editor allows you to write different scripting languages as well, you should first select the language you are going to use.  
From script editor's own menu, select \ijmenu{[Language -> IJ1 Macro]}.

Then, write your first macro as shown below (see also figure \ref{fig_ScriptEditor}). \\