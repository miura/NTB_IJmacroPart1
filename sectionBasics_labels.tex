\section{Basics}
\label{sec:ImageJMacroBasics}

\subsection{``Hello World!''}
We first try writing a simple macro that prints ''Hello World!'' in the log window of ImageJ. For this, we use a text editor that comes with Fiji, called ``script editor''. 
It has some convenient features such as automatic coloring of macro functions {In programming world, we call this feature ``syntax highlighter''}.

If you are not using Fiji and using native ImageJ, it's no problem as there is a simpler but perfectly working text editor. Macros we write in this textbook works exactly same in both editors. If you want to use the simple text editor for the following tutorial, its usage is explained after the explanation about Fiji edior (page \pageref{part:nativeeditor}).

Let's open the ``script editor'' 
by \ijmenu{[File -> New -> Script]}. It should look like figure \ref{fig_ScriptEditor}.
There should be a blank text field where you write your macro. Since the editor allows you to write different scripting languages as well, you should first select the language you are going to use.  
From script editor's own menu, select \ijmenu{[Language -> IJ1 Macro]}.

Then, write your first macro as shown below (see also figure \ref{fig_ScriptEditor}). \\

\begin{lstlisting}[numbers=none]
print("Hello World!");

\end{lstlisting}

Don't ignore quotation marks, parenthesis and the semi-colon! 
Syntax highlighter offers automatic coloring of ImageJ functions, because you selected the language "IJ1 macro" in above. It increases the readability of codes.

Then in the bottom-left corner of the script editor, there is a button labeled "Run". Clicking this, you will see that a log window is created (if it is already there, then it will have a new line) printing "Hello World!" (Figure \ref{fig_HelloWorldLog}). Another way to run the macro is via Script Editor menu,  \ijmenu{[Run -> Run]} . You could use Ctrl-R (Windows) or Command-R (OSX) as well.

Later when you want to start writing another macro, you could just create a new tab by \ijmenu{[File > New]} and then select \ijmenu{[Language -> IJ Macro]} again.