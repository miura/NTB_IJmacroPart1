\textbf{NOTE: Simple Macro Editor in native ImageJ}
\label{part:nativeeditor}

If you are using native ImageJ downloaded from NIH site, the the macro editor launches by selecting \ijmenu{[PlugIns -> New -> Macro]} from the menu . Please then write:

\verb+print("Hello World!");+

From the menu of the macro editor (in OSX, the menu switches to the editors own menu when the editor window is active), select \ijmenu{[Macros > Run Macro]}. You should then see "Hello World!" printed in the log window.

The macro editor has simple debugger function, which is not present in Fiji script editor. Debugger assists you to correct mistakes in the code. ImageJ Macro can be written in any text editor such as "Notepad" in Windows but of course there is no debugger function available in this case.