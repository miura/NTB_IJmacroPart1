\textbf{sourcecode} : \href{http://www.example.com/contents}{code/code03.ijm}

The above operation concatenates content of \ilcom{text2} to the content of \ilcom{text1} and produces a third variable \ilcom{text3} that holds the result of concatenation. It should be noted here, that macro has two ways of usage for \ilcom{+}. What we tested in above is ``concatenation''. Another usage is ``addition'' in the next section.

\begin{indentexercise}
{1}
\item Add more string variables and make a longer sentence.\\
\item \textbf{Answer}: One example could be as shown in the figure \ref{var_stringconcat}.

\end{indentexercise}